\documentclass[letterpaper,8pt]{article}

\usepackage{latexsym}
\usepackage[empty]{fullpage}
\usepackage{titlesec}
\usepackage{marvosym}
\usepackage[usenames,dvipsnames]{color}
\usepackage{verbatim}
\usepackage{enumitem}
\usepackage[pdftex]{hyperref}
\usepackage{fancyhdr}


\pagestyle{fancy}
\fancyhf{} % clear all header and footer fields
\fancyfoot{}
\renewcommand{\headrulewidth}{0pt}
\renewcommand{\footrulewidth}{0pt}

% Adjust margins
\addtolength{\oddsidemargin}{-0.375in}
\addtolength{\evensidemargin}{-0.375in}
\addtolength{\textwidth}{1in}
\addtolength{\topmargin}{-.5in}
\addtolength{\textheight}{1.0in}

\urlstyle{same}

\raggedbottom
\raggedright
\setlength{\tabcolsep}{0in}

% Sections formatting
\titleformat{\section}{
  \vspace{-4pt}\scshape\raggedright\large
}{}{0em}{}[\color{black}\titlerule \vspace{-5pt}]

%-------------------------
% Custom commands
\newcommand{\resumeItem}[1]{
  \item\small{
    {#1}
  }
}

\newcommand{\resumeSubheading}[4]{
  \vspace{-1pt}\item
    \begin{tabular*}{0.97\textwidth}{l@{\extracolsep{\fill}}r}
      \textbf{#1} & #2 \\
      \textit{\small#3} & \textit{\small #4} \\
    \end{tabular*}\vspace{-5pt}
}

\newcommand{\resumeSubItem}[2]{\resumeItem{#1}{#2}\vspace{-1pt}}

\renewcommand{\labelitemii}{$\circ$}

\newcommand{\resumeSubHeadingListStart}{\begin{itemize}[leftmargin=*]}
\newcommand{\resumeSubHeadingListEnd}{\end{itemize}}
\newcommand{\resumeItemListStart}{\begin{itemize}}
\newcommand{\resumeItemListEnd}{\end{itemize}\vspace{-5pt}}

%-------------------------------------------
%%%%%%  CV STARTS HERE  %%%%%%%%%%%%%%%%%%%%%%%%%%%%


\begin{document}

%----------HEADING-----------------
\begin{center} {\Huge \textmd{Anirudh Prabhakaran}} \end{center}
\begin{center}
    \href{mailto:anirudhprabhakaran3@gmail.com}{\underline{Mail}} -
    \href{https://www.linkedin.com/in/anirudhprabhakaran/}{\underline{LinkedIn}} -
    \href{https://github.com/anirudhprabhakaran3}{\underline{GitHub}} -
    \href{https://twitter.com/anirudh23p}{\underline{Twitter}} -
    \href{https://instagram.com/anirudhprabhakaran}{\underline{Instagram}}
  
\end{center}


%-----------EDUCATION-----------------
\section{Education}
  \resumeSubHeadingListStart
    \resumeSubheading
      {National Institute of Technology Karnataka, Surathkal}{December 2020 -  May 2024 \emph{(Expected)}}
      {Bachelor of Technology }{Major CGPA: 9.13}
      \newline
      \newline
      {Major: Electronics and Communication Engineering, Minor: Computer Science}
    \resumeSubheading
      {Sardar Patel Vidyalaya, New Delhi}{2020}
      {XII - CBSE}{XII: 97.80\%}
  \resumeSubHeadingListEnd


%-----------EXPERIENCE-----------------
\section{Experience}
  \resumeSubHeadingListStart

    \resumeSubheading{
    Research Internship
    } {February 2023 - Present}
    {Under the guidance of Dr. Jeny Rajan, Dept. of CSE, NITK}{}

          \resumeItemListStart
           \resumeItem
          {Started working on a project on Multilabel Classification of Hemorrhage Subclasses}
          \resumeItem
          {Tech Stack: Pandas, NumPy, PyTorch}
      \resumeItemListEnd
    
  
          \resumeSubheading
      {\href{https://summerofcode.withgoogle.com/programs/2022/projects/njZwvRi0}{\underline{Google Summer of Code (GSoC) Contributor}}}{June 2022 - August 2022}
      {\href{https://github.com/publiclab/plots2}{\underline{Public Lab}}}{}
      \resumeItemListStart
           \resumeItem
          {Worked on deprecating legacy code to improve the performance of the application.}
          \resumeItem
          {Reduced a few DB query times to 50\%, for a userbase of 500k+ users.}
        \resumeItem
        {Tech Stack: Ruby on Rails, React, MySQL}
      \resumeItemListEnd
  
        \resumeSubheading
      {Machine Learning Intern}{May 2022 - June 2022}
      {\href{https://in.fourthfrontier.com/}{\underline{Fourth Frontier}}}{}
      \resumeItemListStart
        \resumeItem
          {Worked on creating a new model based on a research paper that aims to determine cardiac age from ECG signals in real-time.}
        \resumeItem
        {
            Several models, including ResNet-18, were repurposed for linear regression problems based on ECG data.
        }
        \resumeItem
        {Tech Stack: Python, PyTorch, Jupyter}
      \resumeItemListEnd
  
      \resumeSubheading
      {Software Engineering Intern}{February 2022 - April 2022}
      {\href{https://ai4bharat.org}{\underline{AI4Bharat}}}{}
      \resumeItemListStart
                 \resumeItem
          {Created a data accumulation and annotation platform, \href{https://github.com/AI4Bharat/Shoonya/}{\underline{\textbf{Shoonya}}}, used by language experts to collect data on Indian languages.}
          \resumeItem{
            Used by around 100 annotators since April 2022, with 120k tasks completed over 24 Indian languages and 3 project types.
          }
        \resumeItem{
            Tech Stack: Python, Django, React, PostgreSQL
        }
      \resumeItemListEnd

    \resumeSubheading
      {IRIS Labs Secretary}{March 2021 - Present}
      {IRIS, NITK}{}
      \resumeItemListStart
        \resumeItem
          {\href{https://iris.nitk.ac.in}{\underline{IRIS}} is the student-led ERP developed for automating all administrative and academic activities. User base of around 19k users}
        \resumeItem
        {Working on Urja (IoT) - a mobile app to facilitate the usage of the e-vehicle charging station installed in NITK.}
        \resumeItem
        {Working on Gyan Summarisation (ML) - consolidating advice for placement season from seniors' reports.}
           \resumeItem
          {Contributed to developing various modules like Hostel and Mess Allotment, Academic Certificates, Career Development Center and Alumni Connect, and the maintenance of many other modules.}
      \resumeItemListEnd

  \resumeSubHeadingListEnd


%-----------PROJECTS-----------------
\section{Projects}
  \resumeSubHeadingListStart
      \resumeSubItem
    {\textbf{\underline{Optic Disc Segmentation and Glaucoma Detection:}} A project to first segment the optic disc from various retinal fundus images, then classify these segmented images into glaucomic and non-glaucomic images. Models like U-Net were used for segmentation; custom CNN, AlexNet and ResNet were used for classification. }
    \resumeSubItem
    {\textbf{\href{https://ieee.nitk.ac.in/virtual-expo/Lung-Disease/}{\underline{Lung Disease Detection: }}}A project to detect lung diseases in patients using chest x-rays through deep learning techniques, more specifically, falls under medical imaging. Multiple models, like AlexNet, VGG16 and ResNets, were used.}

  \resumeSubHeadingListEnd

%
%--------PROGRAMMING SKILLS------------
\section{Technical Skills}
 \resumeSubHeadingListStart
 
 \item{
    Python, SQL, Flask, TensorFlow, Scikit-Learn, PyTorch, Keras, React, Golang, Docker, ELK Stack, Django, Ruby on Rails
 }
   
 \resumeSubHeadingListEnd

%-------------------------------------------
\section{Club Activities}
 \resumeSubHeadingListStart
    \item{
     { \textbf{Treasurer, IEEE-NITK} }
   }
   \item {
    Mentored 13 students in field of Machine Learning and Web Development as part of IEEE Mentorship Programs
   }
 \resumeSubHeadingListEnd


\end{document}
